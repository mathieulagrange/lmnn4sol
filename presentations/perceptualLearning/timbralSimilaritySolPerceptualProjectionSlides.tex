
\documentclass{beamer}
 \usepackage{beamerthemedefault, multimedia}
 \useoutertheme{smoothbars}
 \useinnertheme[shadow=true]{rounded}
 \setbeamercovered{transparent}
 \setbeamertemplate{navigation symbols}{}
 \setbeamertemplate{footline}[frame number]
\usepackage{graphicx}
\usepackage{morefloats}
\usepackage{amsmath}
\usepackage{amssymb}
\usepackage{rotating}
% mcode options for matlab code insertion bw (for printing), numbered (line numbers), framed (frame around code blocks), useliterate (convert Matlab expressions to Latex ones), autolinebreaks (automatic code wraping, use it with caution
\graphicspath{{../../experiments/timbralSimilaritySol/report/tex/}{figures/}{tex/}{../figures/}{../../}{../}}
\title{timbralSimilaritySolPerceptualProjectionSlides}
\author{ Mathieu Lagrange }

\begin{document}

\maketitle

\begin{frame}\frametitle{Rationale}
\begin{itemize}
  \item Question: can human judgments that involve percpetion and cognition be approximated by computational means ?
  \item Domain: appreciation of closeness of musical instruments sounds
  \item Proposition: some primary aspects (perception) can be encoded with fixed tranformation of the data, while others (cognition) needs to be encoded using a learning strategy
  \item Implementation: time / frequency wavelet scattering as perceptual "fixed" layer, large-margin nearest neighbors (lmnn) projection for "adaptive" cognitive layer.
\end{itemize}
\end{frame}

\begin{frame}\frametitle{Perceptual judgments}
\begin{itemize}
  \item selected 78 sounds that are "close" to other sounds played by a different instrument
  \item ask human subjects to organize spatially and cluster (by color) those 78 sounds
  \item 31 reference clustering
\end{itemize}
\end{frame}

\begin{frame}\frametitle{Audio data}
\begin{itemize}
  \item
  \item
  \item
\end{itemize}
\end{frame}

\begin{frame}\frametitle{Procesing chain}
\begin{itemize}
  \item Features: mfccs, time frequency wavelet scattering
  \item Projection: lmnn using several flavors
  \item Performance: average p@k for the differents judgments clusters
\end{itemize}
\end{frame}

\begin{frame}\frametitle{Cluster ensemble}
\begin{itemize}
  \item
  \item
  \item
\end{itemize}
\end{frame}

\begin{frame}\frametitle{}
\begin{itemize}
  \item
  \item
  \item
\end{itemize}
\end{frame}



\begin{frame}\frametitle{Factors flow graph}
\begin{center}
\begin{figure}
\includegraphics[width=\textwidth,height=0.8\textheight,keepaspectratio]{../figures/factors.pdf}
\label{factorFlowGraph}
\end{figure}
\end{center}
\end{frame}

\begin{frame}\frametitle{sct: 1000, split: none, reference: judgments, randomize: 0, expand: 0, cut: 1, median: 1, compress: 1, standardize: 1}

\begin{table}
\begin{center}
\
 \setlength{\tabcolsep}{.16667em}
\begin{tabular}{llllc}
features & projection & averageJudgment & separateJudgment & p (\%) \\
\hline
mfcc & lmnn & 0 & 0 &  86.31 $\pm$5.91 \\
mfcc & lmnn & 0 & 1 &  86.18 $\pm$6.05 \\
mfcc & lmnn & 1 & 0 &  86.22 $\pm$5.92 \\
mfcc & none & 1 &  &  85.07 $\pm$6.19 \\
mfcc & lda & 1 &  &  81.50 $\pm$7.65 \\
scat & lmnn & 0 & 0 &  93.31 $\pm$3.92 \\
scat & lmnn & 0 & 1 & \textbf{\textcolor{red}{ 98.09 $\pm$1.28}} \\
scat & lmnn & 1 & 0 &  94.80 $\pm$3.26 \\
scat & none & 1 &  &  87.01 $\pm$5.81 \\
scat & lda & 1 &  & 80.95 $\pm$10.37 \\
\end{tabular}
\end{center}
\label{sc1000SpnoRejuRa0Ex0Cu1Me1Co1St1}
\end{table}
\end{frame}

\begin{frame}\frametitle{\small split: none, reference: judgments, randomize: 0, expand: 0, cut: 1, median: 1, compress: 1, standardize: 1}
\begin{center}
\begin{figure}
\centering
\includegraphics[width=\textwidth,height=0.8\textheight,keepaspectratio]{./figures/Fig160.pdf}
\label{spnoRejuRa0Ex0Cu1Me1Co1St1}
\end{figure}
\end{center}
\end{frame}

\end{document}
